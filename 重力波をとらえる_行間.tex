\documentclass[autodetect-engine,dvi=dvipdfmx,ja=standard, 10pt, a4paper]{bxjsarticle}

\usepackage{amsmath}
\usepackage{mathtools}
\usepackage{amssymb}
\usepackage{physics}
\usepackage{graphicx}
\usepackage{bm}
\usepackage{hyperref}

\title{重力波をとらえる 行間}
\author{江上大晴}

\begin{document}
\maketitle

\section*{第一章 序論}
\subsection*{1-1 宇宙の諸階層}
\subsection*{1-2 ニュートン力学と宇宙の諸階層}
\subsection*{1-3 初期値問題としてのアインシュタイン方程式}
アインシュタイン方程式は
\begin{align}
	G_{\mu\nu} = R_{\mu\nu} -\frac{1}{2}g_{\mu\nu}R = \frac{8\pi G}{c^4}T_{\mu\nu}
\end{align}
ここで、$G_{\mu\nu}$はアインシュタインテンソル、$R_{\mu\nu}$はリッチテンソル、$R$はスカラー曲率、$g_{\mu\nu}$は計量テンソル、$T_{\mu\nu}$は物質のエネルギー運動量テンソル、$G$は万有引力定数、$c$は光速である。

ここで、$G=c=1$の単位系を用いると、アインシュタイン方程式は
\begin{align}
	G_{\mu\nu} = R_{\mu\nu} -\frac{1}{2}g_{\mu\nu}R = 8\pi T_{\mu\nu}
\end{align}
となる。

今回は、計量テンソルを
\begin{align}
	g_{\mu\nu} &= \pmqty{ -\alpha^2 + \beta_i\beta^i & \beta_j\\ \beta_i & \gamma_{ij} }\\
	\beta_i &= \gamma_{ij}\beta^j
	\label{eq:metric}
\end{align}
と表現する。このとき、$\alpha$はラプス関数、$\beta^i$はシフトベクトル、$\gamma_{ij}$は空間計量である。(本文参照)

このとき、反変計量テンソル$g^{\mu\nu}$は
\begin{align}
	g^{\mu\nu} = \pmqty{ -\dfrac{1}{\alpha^2} & \dfrac{\beta^j}{\alpha^2} \\ \dfrac{\beta^i}{\alpha^2} & \gamma^{ij} - \dfrac{\beta^i\beta^j}{\alpha^2} }
	\label{eq:invmetric}
\end{align}

ここで、$\gamma^{ij}$は$\gamma_{ij}$の逆行列である。
つまり、
\begin{align}
	\gamma^{ik}\gamma_{kj} = \delta^i_j
\end{align}

また、$\beta^i =0$とすると、空間座標一定の曲線が法線と一致するように空間座標を選んだことになる。規格化$(n^\mu n_\mu=-1)$された法線ベクトルは
\begin{align}
	n_\mu &= (-\alpha, 0, 0, 0)\label{eq:nmudown}\\
  n^\mu &= \qty(\dfrac{1}{\alpha}, -\dfrac{\beta^i}{\alpha})\label{eq:nmuup}
\end{align}
となる。

今、時間方向と空間方向を明確に区別するために、空間方向への射影テンソルを
\begin{align}
	P_{\mu\nu} = g_{\mu\nu} + n_\mu n_\nu\label{1}
\end{align}
と定義する。あるいは、
\begin{align}
	P^\mu_\nu = g^{\mu\alpha}P_{\alpha\nu} = \delta^\mu_\nu + n^\mu n_\nu
\end{align}
で定義する。

空間方向への射影テンソルとは、時間方向への成分、つまり$n^\mu$の成分を取り除く働きをするテンソルである。確かめると、
\begin{align}
	P_{\mu\nu}n^\nu &= \qty(g_{\mu\nu} + n_\mu n_\nu)n^\nu = n_\mu + n_\mu (n_\nu n^\nu) = n_\mu - n_\mu = 0,
	\label{eq:kuukan_zikan}\\
	P^\mu_\nu n^\nu &= \qty(\delta^\mu_\nu + n^\mu n_\nu)n^\nu = n^\mu + n^\mu (n_\nu n^\nu) = n^\mu - n^\mu = 0  \label{eq:kuukan_zikan_up}
\end{align}
より確かめられる。

この射影テンソルを用いて、任意のベクトル$V_\mu$を空間方向に射影すると
\begin{align}
	\tilde{V}_\mu = P_\mu^\nu V_\nu
\end{align}
の様に定義すると、確かに
\begin{align}
	n^\mu \tilde{V}_\mu = n^\mu P_\mu^\nu V_\nu = 0
\end{align}
の様に時間方向の成分が取り除かれていることが分かる。

任意のテンソルに関しては
\begin{align}
	\tilde{T}_{\mu\nu} = P_\mu^\alpha P_\nu^\beta T_{\alpha\beta}
\end{align}
の様に定義される。

例として$n_\mu$の共偏微分$n_{\mu;\nu}$を考える。このテンソルを空間方向に射影したものを第二基本形式または、外部曲率$K_{ij}$と呼ぶ。共偏微分の定義と外部曲率の定義は
\begin{align}
	n_{\mu;\nu} &= \pdv{n_\mu}{x^\nu} - \Gamma^\alpha_{\mu\nu}n_\alpha\\
	K_{ij} &= - P_i^\mu P_j^\nu n_{\mu;\nu}
\end{align}

次の計算のために使うクリストッフェル記号の定義は
\begin{align}
	\Gamma^\alpha_{\mu\nu} &= \frac{1}{2}g^{\alpha\beta}\qty(\pdv{g_{\beta\nu}}{x^\mu} + \pdv{g_{\beta\mu}}{x^\nu} - \pdv{g_{\mu\nu}}{x^\beta})\\
	\Gamma_{\alpha,\mu\nu} &= g_{\alpha\beta}\Gamma^\beta_{\mu\nu} = \frac{1}{2}\qty(\pdv{g_{\alpha\nu}}{x^\mu} + \pdv{g_{\alpha\mu}}{x^\nu} - \pdv{g_{\mu\nu}}{x^\alpha})
\end{align}

ここで式\eqref{eq:metric}、\eqref{eq:nmudown}式\eqref{eq:nmuup}と直前の4式を用いて、外部曲率$K_{ij}$を計算すると
\begin{align}
	K_{ij} &= -(\delta_i^\mu + n^\mu n_i)(\delta_j^\nu + n^\nu n_j)n_{\mu;\nu}\nonumber\\
	&= -(\delta_i^\mu \delta_j^\nu + n^\mu n_i \delta_j^\nu + \delta_i^\mu n^\nu n_j + n^\mu n_i n^\nu n_j)n_{\mu;\nu}\nonumber\\
	&= -\delta_i^\mu \delta_j^\nu n_{\mu;\nu} - n^\mu n_i \delta_j^\nu n_{\mu;\nu} - \delta_i^\mu n^\nu n_j n_{\mu;\nu} - n^\mu n_i n^\nu n_j n_{\mu;\nu}\nonumber\\
	&= n_{i;j}\nonumber\\
	&= -\Gamma^\alpha_{ij}n_\alpha\nonumber\\
	&= -\Gamma^\alpha_{ij}g_{\alpha\beta}n^\beta\nonumber\\
	&= -\Gamma_{\beta,ij}n^\beta\nonumber\\
	&= \qty(\frac{1}{\alpha}\Gamma_{0,ij} - \frac{\beta^k}{\alpha}\Gamma_{k,ij})\nonumber\\
	&= \frac{1}{2\alpha}\qty( \pdv{g_{0j}}{x^i} + \pdv{g_{0i}}{x^j} - \pdv{g_{ij}}{t} - 2\beta^k \Gamma_{k,ij} )\nonumber\\
	&= \frac{1}{2\alpha}\qty( \qty(\pdv{\beta_i}{x^j} + \pdv{\beta_j}{x^i} - \pdv{\gamma_{ij}}{t}) - 2\beta^k \Gamma_{k,ij} )\nonumber\\
	&= \frac{1}{2\alpha}\qty( \qty(\pdv{\beta_i}{x^j} - \Gamma^k_{ij}\beta_k) + \qty(\pdv{\beta_j}{x^i} - \Gamma^k_{ji}\beta_k) - \pdv{\gamma_{ij}}{t} )\nonumber\\
	&= \frac{1}{2\alpha}\qty(-\pdv{\gamma_{ij}}{t} + \beta_{i|j} + \beta_{j|i})
\end{align}
と計算できる。この式は普通
\begin{align}
	\pdv{\gamma_{ij}}{t} = -2\alpha K_{ij} + \beta_{i|j} + \beta_{j|i}\label{eq:gamma_evolution}
\end{align}
と書くことが多いらしい。

さて、アインシュタイン方程式を初基地問題として考えるためには、アインシュタイン方程式を時間方向と空間方向に射影することを考える。射影の仕方は3通りあるらしい。添え字をそれぞれ1)両方時間方向、2)片方時間方向、3)両方空間方向である。それぞれに対応して
\begin{align}
	1) \quad & G_{\mu\nu}n^\mu n^\nu = 8\pi \rho_H = 8\pi T_{\mu\nu}n^\mu n^\nu\label{eq:energy_constraint}\\
	2) \quad & G_{\mu\nu}n^\mu P^\nu_i = -8\pi J_i = 8\pi T_{\mu\nu}n^\mu P^\nu_i\label{eq:momentum_constraint}\\
	3) \quad & G_{\mu\nu}P^\mu_i P^\nu_j = 8\pi S_{ij} = 8\pi T_{\mu\nu}P^\mu_i P^\nu_j\label{eq:space_space}
\end{align}
これらの変数はそれぞれ、$\rho_H$はエネルギー密度、$J_i$は運動量密度、$S_{ij}$は圧力テンソルである。

さて、それぞれの式を$\alpha$、$\beta^i$、$\gamma_{ij}$で表現しようと頑張って計算すると次のようになる。(付録b参照)
\begin{align}
	1) \quad & ^{(3)}R + K^2 - K_{ij}K^{ij} = 16\pi \rho_H
	\label{eq:hamiltonan_kousoku}\\
	2) \quad & K^j_{i|j} - K_{|i} = 8\pi J_i
	\label{eq:momentum_kousoku}
\end{align}
\begin{align}
	3) \quad  \pdv{K_{ij}}{t} = & \alpha \qty(^{(3)}R_{ij} + K K_{ij} ) -2\alpha K_{il}K^l_j\nonumber\\
	& -8\pi\alpha\qty( S_{ij} + \frac{1}{2}\gamma_{ij}(\rho_H-S^l_l)) -\alpha_{|ij} \\
	& + \beta^m_{|j} K_{mi} + \beta^m_{|i} K_{mj} + \beta^m K_{ij|m}
	\label{eq:evolution_equation}
\end{align}
ここで式\eqref{eq:hamiltonan_kousoku}はハミルトニアン拘束方程式、式\eqref{eq:momentum_kousoku}は動量拘束方程式、式\eqref{eq:hamiltonan_kousoku}から式\eqref{eq:evolution_equation}はアインシュタイン方程式の(3+1)次元形式と呼ぶ。
ここで、式\eqref{eq:hamiltonan_kousoku}は1本の式、式\eqref{eq:momentum_kousoku}は3本の式、式\eqref{eq:evolution_equation}は6本の式からなり、合わせて10本の方程式になる。アインシュタイン方程式も10本の方程式であったので、対応していることが分かる。
空間的な式\eqref{eq:hamiltonan_kousoku}と式\eqref{eq:momentum_kousoku}いて、時間的な式\eqref{eq:evolution_equation}と式\eqref{eq:gamma_evolution}を用いて時間発展を計算すると初期値問題としてアインシュタイン方程式を解いたことになる、

また、流体力学の方程式がどのように表現されているのかを確認する。ビアンキ恒等式$G_{\mu}^{\nu}{}_{;\nu}=0$より、流体の運動式
\begin{align}
	T_{\mu}^{\nu}{}_{;\nu} = 0\label{eq:fluid_equation_tensor}
\end{align}
が導かれる。ただし、ここでの$G$はアインシュタインテンソルで時空の曲がり具合を表し、$T$はエネルギー運動量テンソルである。ビアンキ恒等式は一般相対論の本を読むと出てくるらしい。

アインシュタイン方程式の(3+1)次元表現では$\rho_H, J_i, S_{ij}$が現れたので流体力学の式をこれらの変数で表現する。エネルギー運動量テンソル$T_{\mu\nu}$は式\eqref{eq:energy_constraint}、\eqref{eq:momentum_constraint}、\eqref{eq:space_space}より
\begin{align}
	T_{\mu\nu}& = g_{\mu\alpha}g_{\nu\beta}T^{\alpha\beta}\nonumber\\
	&=(P_{\mu\alpha} - n_\mu n_\alpha)(P_{\nu\beta} - n_\nu n_\beta)T^{\alpha\beta}\nonumber\\
	&=P_{\mu\alpha}P_{\nu\beta}T^{\alpha\beta} - n_\mu n_\alpha P_{\nu\beta}T^{\alpha\beta} - n_\nu n_\beta P_{\mu\alpha}T^{\alpha\beta} + n_\mu n_\nu n_\alpha n_\beta T^{\alpha\beta}\nonumber\\
	&= S_{\mu\nu} + n_\mu J_\nu + n_\nu J_\mu +\rho_H n_\mu n_\nu 
\end{align}
がわかる。

ここで、エネルギー運動量テンソルを$n^\mu$方向(時間方向)と$P_i^\mu$方向に射影して式\eqref{eq:fluid_equation_tensor}を表現する。まず、時間方向は
\begin{align}
	n^\mu T_{\mu}^{\nu}{}_{;\nu} = 0
\end{align}
であるので、
\begin{align}
	0 = (n^\mu T_{\mu}^{\nu})_{;\nu} - T_{\mu}^{\nu} n^\mu{}_{;\nu} 
\end{align}
まず、$n^\mu T_{\mu}^{\nu}$を計算すると式\eqref{eq:momentum_constraint}、\eqref{eq:space_space}と式\eqref{eq:kuukan_zikan}より、$n^\mu S_{\mu}^{\nu} = 0$、$n^\mu J_\mu = 0$であることを用いると
\begin{align}
	n^\mu T_{\mu}^{\nu} &= n^\mu \qty( S_{\mu}^{\nu} + n_\mu J^{\nu} + n^{\nu} J_\mu + \rho_H n_\mu n^{\nu} )\nonumber\\
	&= n^\mu S_{\mu}^{\nu} + n^\mu n_\mu J^{\nu} + n^{\nu} n^\mu J_\mu + \rho_H n^\mu n_\mu n^{\nu} \nonumber\\
	&= - J^{\nu} -\rho_H n^{\nu}
\end{align}

これを共変微分するが求めたいのはその発散であるから、一般的なベクトルの発散の式を示すと
\begin{align}
	A^\mu{}_{;\mu} = \frac{1}{\sqrt{-g}}\pdv{x^\mu}\qty(\sqrt{-g}A^\mu)
\end{align}
となるということを用いる。(ここは頑張れ)ここで$g$は計量テンソルの行列式である。これを求めると
\begin{align}
	g &= \det(g_{\mu\nu}) \nonumber\\
	&= \det\pmqty{ -\alpha^2 + \beta_i\beta^i & \beta_j\\ \beta_i & \gamma_{ij} } \nonumber\\
	&= -\alpha^2 \det(\gamma_{ij}) \nonumber\\
	&= -\alpha^2 \gamma
\end{align}
と分かるのでこれらを代入すると
\begin{align}
	(n^\mu T_{\mu}^{\nu})_{;\nu} &= \frac{1}{\sqrt{-g}}\pdv{x^\nu}\qty(\sqrt{-g}n^\mu T_{\mu}^{\nu})\nonumber\\
	&= \frac{1}{\alpha\sqrt{\gamma}}\pdv{x^\nu}\qty(\alpha\sqrt{\gamma}(- J^{\nu} -\rho_H n^{\nu}))\nonumber\\
	&= -\frac{1}{\alpha\sqrt{\gamma}}\pdv{x^\nu}\qty(\alpha\sqrt{\gamma} J^{\nu}) - \frac{1}{\alpha\sqrt{\gamma}}\pdv{x^\nu}\qty(\alpha\sqrt{\gamma}\rho_H n^{\nu})\nonumber\\
	&= -\frac{1}{\alpha\sqrt{\gamma}}\pdv{x^\ell}\qty(\alpha\sqrt{\gamma} J^{\ell}) + \frac{1}{\alpha\sqrt{\gamma}}\pdv{t}\qty(\sqrt{\gamma}\rho_H) + \frac{1}{\alpha\sqrt{\gamma}}\pdv{x^\ell}\qty(\sqrt{\gamma}\rho_H \beta^\ell)
\end{align}
が分かる。

また、ここで完全流体を想定することなどにより
\begin{align}
	T_{\mu\nu} = (\rho(1+\epsilon) + p)u_\mu u_\nu + p g_{\mu\nu}\\
	V^{\ell} = \frac{\alpha J^{\ell}}{\rho_H + p} - \beta^{\ell} = \frac{u^{\ell}}{u^0}
\end{align}
と定義して考える、すると、$T_{\mu}^{\nu} n^\mu{}_{;\nu}$は
・・・

付録b参照

\begin{align}
	&\pdv{t}\qty(\sqrt{\gamma}\rho_H) + \pdv{x^l}\qty(\sqrt{\gamma}\rho_H V^l)\nonumber\\
	&=-\pdv{x^l}\qty(\sqrt{\gamma}p(V^l+\beta^l)) +\alpha\sqrt{\gamma}pK - \pdv{\alpha}{x^l}\sqrt{\gamma}J^l+\frac{\alpha\sqrt{\gamma}J^lJ^m K_{lm}}{\rho_H + p}
\end{align}


\section*{第二章 重力波}
\subsection*{2-1 重力波の整形理論}
全勝で初基地問題として考えたアインシュタイン方程式の中には4つの自由度が存在する。ここでは、その自由度を取り出す。物質がないとして$T_{\mu\nu}=0$とすると、アインシュタイン方程式は
\begin{align}
	R_{\mu\nu} - \frac{1}{2}g_{\mu\nu}R = 0
\end{align}
となるいま、計量テンソル$g_{\mu\nu}$がミンコフスキー空間の計量$\eta_{\mu\nu}$から小さくずれたものであると仮定する。つまり、
\begin{align}
	g_{\mu\nu} = \eta_{\mu\nu} + h_{\mu\nu}
\end{align}
とする。ここでは$\epsilon$の一次に比例する量だけ取り扱う線形理論を展開する。

アインシュタイン方程式を$h_{\mu\nu}$で表現するために、リーマンテンソルを考えると
\begin{align}
	R_{\mu\nu\alpha\beta} &= \frac{1}{2}\qty(\pdv{^2 g_{\mu\beta}}{x^\nu \partial x^\alpha} + \pdv{^2 g_{\nu\alpha}}{x^\mu \partial x^\beta}  - \pdv{^2 g_{\nu\beta}}{x^\mu \partial x^\alpha}- \pdv{^2 g_{\mu\alpha}}{x^\nu \partial x^\beta}) + (\Gamma_{\rho,\nu\beta}\Gamma^\rho_{\mu\alpha} - \Gamma_{\rho,\mu\beta}\Gamma^\rho_{\nu\alpha})
\end{align}
ここで、クリストッフェル記号は$\epsilon$の1次のオーダーなので、クリストッフェルの2次のオーダーの項は無視できる。よって、
\begin{align}
	R_{\mu\nu\alpha\beta} {}^{(1)}&= \frac{\epsilon}{2}\qty(h_{\mu\beta,\nu\alpha} + h_{\nu\alpha,\mu\beta} - h_{\nu\beta,\mu\alpha}- h_{\mu\alpha,\nu\beta})
\end{align}



\end{document}